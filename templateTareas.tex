\documentclass[11pt]{article}
\usepackage[margin=1in]{geometry}
\usepackage{amsthm}
\usepackage{hyperref}
\hypersetup{
    colorlinks = true
}
\usepackage{newtxtext,newtxmath}
\usepackage[spanish]{babel}
\usepackage{amsmath}
\usepackage{tikz}
\usepackage[latin1]{inputenc}
\usepackage[T1]{fontenc}

\usepackage{algorithm}
\usepackage[noend]{algpseudocode}
\makeatletter
\renewcommand{\ALG@name}{Algoritmo}
\makeatother


\usepackage{array}
\newcolumntype{L}[1]{>{\raggedright\let\newline\\\arraybackslash\hspace{0pt}}m{#1}}

\usepackage{pgfplots}
\pgfplotsset{compat=1.9}

%%%%%% Aqu� comienza el documento %%%%%%%%


\title{Dise�ar algoritmos es genial: un estudio pr�ctico}
\author{Persona 1 \and Persona 2 \and Persona 3}
\begin{document}
\maketitle

\begin{abstract}
	\href{http://comunicacionacademica.uc.cl/images/recursos/ingles/lectura_esencial/pdf/2_lee_el_abstract.pdf}{Aqu� un enlace sobre como escribir res�menes para art�culos}. A pesar de los claros argumentos te�ricos que establecen la disciplina del dise�o y an�lisis de algoritmos como una de las m�s divertidas ocupaciones humanas, es dif�cil encontrar experimentos que validen tales conclusiones te�ricas en la pr�ctica. En este art�culo discutimos entrevistas a Pedro, Juan y Diego, a trav�s de un cuidadoso an�lisis que nos permite concluir que efectivamente los algoritmos son m�s entretenidos que el f�tbol. 
\end{abstract}


\section{Introducci�n}

Aqu� va el problema, los algoritmos a estudiar, las m�tricas a estudiar, la notaci�n, etc.

\begin{equation}
	\log \log (n) ^ {\log \log (n)} \in O(n) 
\end{equation}

\begin{algorithm}
\caption{Un algoritmo uwu}
\label{alg:uwu}
\begin{algorithmic}[1]
\Procedure{DoUwu}{}
\State $\textit{numberUwus} \gets 10$
\For{$i \in \{1, \ldots, \textit{numberUwus}\}$}
  \State \textbf{print} ''uwu''
\EndFor
\Return owo
\EndProcedure
\end{algorithmic}
\end{algorithm}



\section{Hip�tesis}
Lo que creen que ocurrir� y por qu�.

\begin{table}[ht]
\begin{tabular}{L{3cm}|L{6cm}|L{6cm}|}
\cline{2-3}
                                    & \# de comparaciones                  & Tiempo de ejecuci�n\\ \hline
\multicolumn{1}{|l|}{Caso Promedio} & Ambas variantes se comportan de manera similar. &  La variante cl�sica es m�s r�pida. \\ \hline
\multicolumn{1}{|l|}{Peor Caso}  & La variante  k-way-BottomUp realiza menos comparaciones.     &  Ambas variantes se comportan de manera similar.      \\ \hline
\end{tabular}
\caption{Hip�tesis sobre la relaci�n entre distintas variantes de Mergesort}
\label{table:hypothesis}
\end{table}

Como se puede observar en el Cuadro \ref{table:hypothesis}, se trata de una tabla.

\section{Dise�o experimental}
Descripci�n de los experimentos a ejecutar, en qu� condiciones, con qu� valores, de d�nde se obtienen los datos o c�mo los generaron, etc.

\section{Resultados}

�Qu� pas�?

\begin{figure}
\centering
\begin{tikzpicture}
\begin{axis}[
%    title={Temperature dependence of CuSO$_4\cdot$5H$_2$O solubility},
    xlabel={Temperature [\textcelsius]},
    ylabel={Solubility [g per 100 g water]},
    xmin=0, xmax=100,
    ymin=0, ymax=120,
    xtick={0,20,40,60,80,100},
    ytick={0,20,40,60,80,100,120},
    legend pos=north west,
    ymajorgrids=true,
    grid style=dashed,
]

\addplot[
    color=blue,
    mark=square,
    ]
    coordinates {
    (0,23.1)(10,27.5)(20,32)(30,37.8)(40,44.6)(60,61.8)(80,83.8)(100,114)
    };
    \legend{CuSO$_4\cdot$5H$_2$O} 
\end{axis}
\end{tikzpicture}
\caption{Resultados experimentales de la solubilidad de algo en funci�n de la temperatura. Confirmando la hip�tesis n�mero 2.}
\label{fig:solubility}
\end{figure}

Como se aprecia en la Figura~\ref{fig:solubility}, al calentar el agua ocurren cosas.

\section{Conclusiones}

�Qu� podemos decir ahora a la luz de lo que ocurri�? (Nota: no es recapitular lo realizado).


%%% Anexo (o ap�ndice)
\appendix
\section{Gr�ficos adicionales}
Anexo solo en caso de ser necesario, por ejemplo para incluir gr�ficos no fundamentales.

\end{document}
